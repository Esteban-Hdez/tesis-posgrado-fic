%----------------------------------------------------------------------------------------
% II. REVISIÓN DE LITERATURA
%----------------------------------------------------------------------------------------

% Subtítulos de segundo nivel (sección)
\section{Nombre de la Sección}
Texto de la sección aquí.

% Subtítulos de tercer nivel (subsección)
\subsection{Nombre de la Subsección}
Texto de la subsección aquí.

% Subtítulos de cuarto nivel (subsubsección)
\subsubsection{Nombre de la Subsubsección}
Texto de la subsubsección aquí.

% Subtítulos de quinto nivel (párrafo)
\paragraph{Nombre del Párrafo}
Texto del párrafo aquí.

% Ejemplo de figura
\begin{figure}[ht]
    \centering
    % Placeholder para la imagen
    \fbox{\parbox{6cm}{\centering Placeholder de la imagen\\[1.5cm]Ejemplo de red neuronal}}
    \caption[Ejemplo de una red neuronal]{Ejemplo de una red neuronal 1.}
    \label{fig:red-neuronal}
\end{figure}

% Ejemplo de algoritmo
\begin{algorithm}
    \caption{Ejemplo de Uso de Red Neuronal Convolucional}
    \begin{algorithmic}[1]
        \Procedure{ClasificaciónConCNN}{$imagen$}
            \State $resultado \gets \text{CNN}(imagen)$
            \State \textbf{return} $resultado$
        \EndProcedure
    \end{algorithmic}
    \label{alg:cnn}
\end{algorithm}

% Ejemplo de ecuación
\begin{equation}
    f(x) = \sigma(W \cdot x + b)
    \label{eq:funcion_activacion}
\end{equation}
\equations{Función de activación sigmoide}

% Para referenciar la ecuación en el texto, usar \ref{eq:funcion_activacion}.

% Ejemplo de cuadro
\begin{table}[ht]
    \centering
    \caption{Ejemplo de un cuadro de datos.}
    \begin{tabular}{lcr}
        \toprule
        Columna 1 & Columna 2 & Columna 3 \\
        \midrule
        Datos 1 & Datos 2 & Datos 3 \\
        Datos 4 & Datos 5 & Datos 6 \\
        \bottomrule
    \end{tabular}
    \label{tab:ejemplo-datos}
\end{table}
% Para referenciar el cuadro en el texto, usar \ref{tab:ejemplo-datos}.

% Ejemplo de como citar
Como se discute en \cite{investigador2023}, las CNN son cruciales para el procesamiento de imágenes.

% Ejemplo de nota al pie de página
Este es un texto con una nota al pie de página\footnote{Aquí va el texto de la nota al pie de página.}.

% Para referenciar figuras, cuadros, algoritmos, ecuaciones en el texto, usar \ref{label_del_elemento}.

